In this work we treat the solvent of a viscoelastic fluid as an incompressible fluid with the usual Navier-Stokes momentum and mass conservation equations. These are given by
\begin{equation}\label{eq:momentum_conservation}
\rho \left( \frac{\partial \bm{u}}{\partial t} + \left(\bm{u} \cdot \nabla\right)\bm{u}\right) =
-\nabla p +\eta_s \nabla^2 \bm{u} + \bm{F}^{\mathrm{ext}} + \nabla \cdot \bm{\tau} 
\end{equation}
and
\begin{equation}\label{eq:mass_conservation}
\nabla \cdot \bm{u} = 0,
\end{equation}
where $\rho$ is the density, $\bm{u}$ is the velocity field, $p$ is the pressure, $\eta_s$ is the viscosity of the solvent, $\bm{F}^{\mathrm{ext}}$ is an external force density on the fluid, and $\bm{\tau}$ the so-called extra stress tensor. Here we see that the divergence of the extra stress tensor leads to a forcing term in the momentum conservation equation. 

Although, on the microscopic level, viscoelastic effects are due to the stretching and relaxation of polymers suspended in the solvent, we will be using a continuum model to develop this extra stress tensor in time. The Oldroyd-B model is commonly used to describe viscoelastic flow and is mathematically equivalent to a fluid filled with harmonic spring "dumbells." The Oldroyd-B model provides a constitutive equation for the extra stress tensor which has the form
\begin{equation}\label{eq:short_oldroyd_b}
\overset{\nabla}{\bm{\tau}} = \frac{1}{\lambda_p}\left(2\eta_p\bm{d}- \bm{\tau}\right)
\end{equation}
where $\lambda_p$ and $\eta_p$ are the relaxation time and added viscosity of the polymer solution, respectively, and  
\begin{equation}\label{eq:rate_of_strain}
\bm{d} \defeq \frac{1}{2}\left(\nabla\bm{u} + \nabla \bm{u}^T\right)
\end{equation}
is the rate of strain tensor where $T$ indicates the transpose. The upper-convected time derivative is defined as
\begin{equation}\label{eq:upper_convected_derivative}
\overset{\nabla}{\bm{\tau}} \defeq \frac{\partial \bm{\tau}}{\partial t} + (\bm{u} \cdot \nabla)\bm{\tau} - \bm{\tau} \cdot \nabla\bm{u} -(\nabla\bm{u})^T \cdot \bm{\tau}
\end{equation}
and describes the time evolution of a tensor being transported along a velocity field $\bm{u}$. Use of this time derivative is required so that our constitutive relation Note that it is important that we choose the convention that
\begin{equation}\label{eq:gradient_convention}
(\nabla \bm{u})_{\alpha\beta} \defeq \frac{\partial u_{\beta}}{\partial x_\alpha}
\end{equation}
in order that Eq.~\eqref{eq:upper_convected_derivative} be the correct form. Finally, we may write the constitutive equation as
\begin{equation}\label{eq:full_oldroyd_b}
\frac{\partial \bm{\tau}}{\partial t} + (\bm{u} \cdot \nabla)\bm{\tau}= \left(\bm{\tau} \cdot \nabla\bm{u} +(\nabla\bm{u})^T \cdot \bm{\tau}\right) + \frac{1}{\lambda_p}\left(2\eta_p \bm{d} - \bm{\tau}\right).
\end{equation}
We aim to develop these two equations numerically in order to simulate an Oldroyd-B fluid.